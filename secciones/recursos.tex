De forma de garantizar la total reproducción de este trabajo con incluso los mismo resultados de análisis independientemente de quien lo realice, donde se lo realice y bajo que requerimientos computacionales, se tomó la decisión de armar un ambiente de trabajo propio y portable, el cual contiene todas las herramientas de software necesarias y configuradas. A su vez, se registra todo avance realizado históricamente con sus modificaciones y código fuente para su ejecución.

Esto se pudo realizar armando previamente al análisis una infraestructura infraestructura. La misma se compone de:

\begin{itemize}
\item \href{https://www.docker.com/}{\textcolor{blue}{\underline{Docker}}}
\item Linux, \href{https://ubuntu.com/}{\textcolor{blue}{\underline{Ubuntu}}}
\item \href{ https://www.r-project.org/}{\textcolor{blue}{\underline{R}}}, \href{https://rstudio.com/}{\textcolor{blue}{\underline{Rstudio}}}
\item \href{https://www.python.org/}{\textcolor{blue}{\underline{Python}}}
\item \href{https://code.visualstudio.com/}{\textcolor{blue}{\underline{Visual Studio Code}}}
\item \href{https://jupyter.org/}{\textcolor{blue}{\underline{Jupyter}}} 
\item \href{https://git-scm.com/}{\textcolor{blue}{\underline{git}}} con metodología \href{https://github.com/nvie/gitflow}{\textcolor{blue}{\underline{gitflow}}}
\item \href{http://projecttemplate.net/index.html}{\textcolor{blue}{\underline{ProjectTemplate}}} 
\end{itemize}

\includegraphics[width=2cm]{imagenes/logos/docker.jpeg}
\includegraphics[width=2cm]{imagenes/logos/ubuntu_logo_2.png}
\includegraphics[width=2cm]{imagenes/logos/Rlogo.png}
\includegraphics[width=2cm]{imagenes/logos/python.jpeg}
\includegraphics[width=2cm]{imagenes/logos/jupyter.png}
\includegraphics[width=2cm]{imagenes/logos/git_logo_2.png}

\vspace{5mm}

\textcolor{red}{
La reproducción es factible si la persona tiene acceso y cuenta con la información de la base de datos con la que se realizó este trabajo que por cuestiones de confidencialidad mencionadas anteriormente no serán entregadas.
}

