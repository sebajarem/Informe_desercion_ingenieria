Exceptuando los modelos no supervisados, todos los modelos empleados han superado el porcentaje que representa la clase mayoritaria, ya sean los modelos que utilizan todos los datos disponibles como aquellos que realizan una selección de las mejores variables e incluso aquellos modelos que se aplicaron eliminando la variable más importante cuya relevancia es notablemente superior a cualquiera de las otras características. Esto da un indicio de que los modelos resultan útiles para cumplir con los objetivos de este análisis.\\

Una característica de los modelos a tener en cuenta es que aún siendo  técnicamente distintos, están dentro de un rango de eficacia cercano. Por lo tanto, si una persona requiere la máxima eficacia y un resultado simple como puede ser una lista de personas que mas probablemente sean desertores, puede optar por la elección del modelo de SVM (en este caso) que obtuvo los mejores resultados pero que no explica de forma directa la relación que tiene con las variables. Por el contrario si la persona que está realizando el análisis, necesita argumentos fehacientes para iniciar algún acción, puede optar por modelos que tienen menos eficacia (no tan lejana del anterior) pero que a su vez explican muy bien las razones de la decisión sobre cada registro. Este es el caso de los árboles de decisión simple o regresión logística.

La variable notablemente más importante en todos los modelos fue "ciclo cursada regular" que contiene el último ciclo lectivo de cursada que tuvo el alumno. Puede ser lógico pensar que un alumno que un ciclo lectivo no cursa es porque ya son desertores. Sin embargo, hubo un tiempo en el que el alumno no podía cursar el año siguiente por mas que tenga las intenciones si es que no cumplía ciertos requisitos. Actualmente esos requisitos son mas flexibles que antes y puede ser una consecuencia de un análisis similar centrándose en un pensamiento: ``los alumnos que se alejan del ámbito universitario son mas propensos a abandonar sus estudios".