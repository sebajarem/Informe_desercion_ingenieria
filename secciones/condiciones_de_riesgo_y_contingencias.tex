% que el estudio lo usen para univerisdaddes de elite o para malos usos


En segundo lugar, no menos importante, es respecto del "buen" o "mal" uso que
pudiera hacerse de la información subyacente y a priori desconocida en los datos
analizados. Por ejemplo, conocer algunas posibles causas de la deserción, así como
el perfil de un potencial desertor puede llevar a un desarrollo de políticas universitarias
que busquen estimular la continuidad en los estudios o por el contrario desalentarla
cuanto antes. Esto puede llevarse a cabo con independencia de si se individualiza
efectivamente o no al estudiante que sea un potencial desertor. Pero si además se lo
identifica, esas políticas podrían ser personalizadas, complementándolas con
entrevistas, cursos de apoyo o asistencia de un tutor. Esto no puede calificarse a priori
como "bueno" o "malo" per se. Basta considerar los problemas de vocación por
ejemplo. Muy probablemente su identificación temprana podría ser beneficiosa para el
estudiante y para la Universidad. Pero también podría aparecer algún tipo de riesgo en
utilizar esos datos para hacer una Universidad elitista.
Más allá de trabajar sobre la posibilidad técnica de encontrar información relevante
subyacente en los datos de los estudiantes está claro que hay que considerar los
aspectos éticos antes de cualquier uso que pudiera dárseles. Es evidentemente un
problema complejo con múltiples aristas y con muchas preguntas por responder.

