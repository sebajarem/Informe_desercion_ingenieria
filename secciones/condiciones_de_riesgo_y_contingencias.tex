% que el estudio lo usen para univerisdaddes de elite o para malos usos



Es importante destacar las consecuencias que pudiera tener el ``buen" o ``mal" uso de la información subyacente y a priori desconocida en los datos analizados. Por ejemplo, conocer algunas posibles causas de la deserción, así como el perfil de un potencial desertor puede llevar a un desarrollo de políticas universitarias
que busquen estimular la continuidad en los estudios o por el contrario desalentarla
cuanto antes. Esto puede llevarse a cabo con independencia de si se individualiza
efectivamente o no al estudiante que sea un potencial desertor. Pero si además se lo
identifica, esas políticas podrían ser personalizadas, complementándolas con
entrevistas, cursos de apoyo o asistencia de un tutor. \\
La identificación temprana de algún patrón o característica particular podría ser beneficiosa para el estudiante y para la Universidad. Sin embargo, también habría que considerar el riesgo que conlleva utilizar esos datos, por ejemplo, convertir la casa de estudios en una Universidad elitista.\\
Más allá de trabajar sobre la posibilidad técnica de encontrar información relevante
subyacente en los datos de los estudiantes está claro que hay que considerar los
aspectos éticos antes de cualquier uso que pudiera dárseles. Es evidentemente un
problema complejo con múltiples aristas y con muchas preguntas por responder.

