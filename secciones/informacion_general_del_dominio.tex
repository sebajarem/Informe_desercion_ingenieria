\subsubsection{La Deserción}\label{seccion:desercion}
En todo el campo educativo, un fenómeno recurrente y de los más relevantes es el de la deserción. Está presente en todos los niveles educativos. Las Universidades no parecen ser la excepción y, de distintas maneras y por distintos motivos, buscaron y buscan una manera de entenderlo. En primer lugar no existe una definición única y que abarque completamente del complejo fenómeno de la deserción.  
Viale Tudela \cite{ErnestoUNAOUT} recoge ciertos intentos de captar estas "intuiciones", en las cuales la idea generalizada apunta hacia el cese de actividades deliberado o forzado del alumno en la institución educadora, pero que no todo cese de actividades representa una
verdadera deserción. En el Glosario de Términos Básicos \cite{CONEAU2010PropuestaSINEACE}, el CONEAU (Perú) recurre a una definición como “proceso de abandono, voluntario o forzoso, de la carrera en la que se matricula un estudiante, por la influencia positiva o negativa de circunstancias internas o externas a él o ella”, pero no suprime la dimensión cuantitativa como una proporción relativa a la duración de las carreras. Aparecen así otros conceptos ligados a la deserción, como la cronicidad y el desgranamiento. El Desgranamiento es el fenómeno por el cual ciertos alumnos de una cohorte -si bien conservan su condición de regularidad- no cumplen con los plazos del plan de estudios de la carrera. La cronicidad describe la situación de alumnos o carreras con altos índices de desgranamiento. Y en general la deserción es el fenómeno por el cual los alumnos de una carrera o de una universidad abandonan el cursado de los estudios y como consecuencia no cumplen con las  condiciones de regularidad.

\subsubsection{Antecedentes}
Dada la importancia de este problema, se han buscado distintos métodos para poder dar con  las causas de la deserción universitaria. Uno de ellos es el de identificar rasgos o variables que puedan ser tomados como indicadores relacionados con la emergencia de la deserción. Los trabajos sobre indicadores académicos son numerosos y buena parte se centra en el rendimiento de los estudiantes e incluyen variables culturales, demográficas, laborales, entre otras, además de su historial académico (materias aprobadas, asignación de becas, repitencia, etc.), aunque existen trabajos más integrales que consideran factores socioeconómicos, personales, académicos e institucionales \cite{L.GonzalezFeigehen2006RepitenciaLatina,SociedadModeloSuperior}.
En ocasiones se realizaron propuestas \cite{Kuna2009IdentificacionUniversitarios} para modificar la emergencia de la deserción incidiendo directamente en las variables tomadas como indicadores.



\subsubsection{Estudio de la Deserción: Enfoque desde la Minería de Datos }\label{seccion:dm_en_desercion}
Dentro del campo de Minería de Datos existen una enorme y muy variada cantidad de modelos de análisis. El``Análisis de Desgaste" es uno de ellos. que nacen por parte de las empresas para detectar los fenómenos que generan el comportamiento de la deserción de sus clientes. Se trata de encontrar la relación entre la deserción y las principales variables que afectan a ello. El objetivo del análisis de desgaste es proporcionar al investigador la capacidad de entender qué variables son las más importante al desgaste y cuál es la probabilidad del abandono del cliente. De esta forma es posible entender el porqué y además predecir qué tipo de clientes son más proclives a la deserción. Estas técnicas se usan actualmente también en al ámbito financiero para evaluar qué personas son buenas/malas pagadoras de créditos, en el ámbito de la salud para evaluar el comportamiento de las personas frente a cierta enfermedad y en las empresas para retener clientes o empleados\cite{Chapman1982AttritionAnalysis}.

