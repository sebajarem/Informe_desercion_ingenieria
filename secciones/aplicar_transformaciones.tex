Se han aplicado varios tipos de transformaciones para armar el tablón general y los datos derivados que se mencionan en la seccion \ref{atributos_derivados}.\\

\begin{itemize}
	\item deserto: la clase ha sido producida a través de transformar la información inicial agrupando y cumpliendo las condiciones señaladas en su definición. \ref{definicion_clase_desertor}
	\item esTecnico: es una variable creada a partir de la variable Estudios.secundarios.
	\item edad al ingreso: resta entre año de nacimiento y año de ingreso a la facultad
	\item el resto de las variables que figuran en el tablón general, son agregaciones de las variables de las tablas de cursada y finales. Se agruparon por alumno y se contaron la cantidad de veces o hicieron promedios máximos y generaron las columnas correspondientes:
	\begin{itemize}
		\item Tipo de aprobación Libre	
		\item Turno Tarde
		\item Tipo de aprobación Cambio Curso
		\item Tipo de aprobación Promociono
		\item Turno Noche
		\item Tipo de aprobación No Firmo
		\item TurnoMañana
		\item Tipo de aprobación Firmo
		\item Cantidad de veces recursada regular
		\item Descripción de recursada regular No Recurso
		\item Descripción de recursada regular Recurso 1 Vez
		\item Descripción de recursada regular Recurso 2 Veces
		\item Descripción de recursada regular Recurso 3 Veces
		\item Descripción de recursada regular Recurso 4 Veces
		\item Descripción de recursada regular Recurso 5 Veces
		\item Descripción de recursada regular Recurso n Veces (mayor a 5 veces )
		\item Aprobado: cantidad de materias aprobadas.
		\item Promociono: cantidad de materias promocionadas.
		\item noAprobado: cantidad de materias no aprobadas
		\item Nota
		\item Nota max prom
	\end{itemize} 
\end{itemize}


