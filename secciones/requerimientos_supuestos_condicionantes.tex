
\subsubsection{Requerimientos}
Es de suma importancia destacar la toma de conciencia para incorporar en las instituciones educativas sistemas que permitan el registro de cada vez más variedad de variables y de distinto tipo respecto de los contextos nombrados anteriormente de los estudiantes.
Esta conformación de bases de datos es imprescindible para este tipo de trabajos, por lo que todo trabajo que esté interesado en realizar un estudio sobre el fenómeno de la deserción, debe contar con ella como si fuera su material de trabajo inicial.
A partir de la toma de conciencia sobre la necesidad de contar con estas complejas bases de datos es que se pueden aplicar las técnicas de Data Mining y es de esperar que se tenga éxito de la misma manera en que se obtienen en otros campos como el de la medicina \cite{Adnan2013DataReview}, marketing, o el bancario\cite{Kirkos2007DataStatements}.


\subsubsection{Condicionantes}\label{seccion:condicionantes}
Ética en el uso responsable de la información y de confidencialidad.
Al tratarse de datos de alumnos de la Universidad, su desenvolvimiento académico, entre otro tipo de información. El GIAR \bnote{GIAR} firmó un acuerdo de confidencialidad para la realización de este trabajo y otros relacionados. A su vez, se tomaron medidas para evitar que los investigadores pudiera identificar a alumnos. Algunas de las medidas adoptadas fueron reemplazar el nombre por un identificador (ID) de cada alumno y que se
eliminen algunos datos personales.
Además, los datos con los que se trabajan son una muestra representativa de todo el universo.
De esta forma, se crea un ambiente de trabajo seguro y anónimo.
