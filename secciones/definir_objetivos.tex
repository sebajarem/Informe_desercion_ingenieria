%# posibles objetivos

\textbf{Objetivo General}\\
\begin{itemize}
	\item El objetivo de este trabajo es ofrecer a los especialistas en educación nuevas herramientas de análisis y descubrimiento de conocimiento sobre el fenómeno de la deserción, y así poder incorporarlas a otros trabajos que busquen modificar la naturaleza y las condiciones de posibilidad del fenómeno.
	\item Inicialmente, se propone analizar fenómenos particulares de deserción, es decir, la deserción en la carrera de ingeniería en sistemas de la UTN-FRBA \bnote{UTN}\bnote{FRBA}.
	\item La intensión también es poder dar el mejor apoyo posible a los alumnos de la Universidad y a la Universidad misma.
\end{itemize}

Por tal motivo, se comenzó con una definición propia, medible y
simple de la situación de deserción definida como: dos años consecutivos sin actividad. Esto es sin cursar ni aprobar finales.


\textbf{Objetivos Particulares}
\begin{itemize}
	\item Desarrollar algoritmos de modelos descriptivos para la explicación de las variables relevantes al problema elegido. Particularmente en este trabajo se estudiará la desersión de los estudiante.
	\item Desarrollar algoritmos de modelos predictivos para evaluar los comportamientos actuales de los estudiantes y brindar los resultados a las personas idóneas para que puedan actuar en consecuencia.
	\item Obtener patrones y evaluar los mismos para crear las reglas y la Base de Conocimiento de un Sistema Experto, mediante el resultado de los patrones obtenidos.
\end{itemize}

\comentarioinvisible{este punto por ahora no lo pongo como objetivo. Desarrollo de un Sistema Experto para evaluar el comportamiento estudiantil a fin de dar soporte a la toma de decisiones.}



