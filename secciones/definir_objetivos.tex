%# posibles objetivos
Objetivo General
El objetivo de este trabajo es ofrecer herramientas para un acercamiento al fenómeno
de la deserción, y así poder incorporarlas a otros trabajos que busquen modificar la
naturaleza y las condiciones de posibilidad del fenómeno. Para no dirigirse a un
objetivo más ambicioso de lo que tal vez pueda resolverse, se propuso analizar
inicialmente fenómenos particulares de deserción, es decir, la deserción en la carrera
de ingeniería en sistemas. Para ello se comenzó con una definición propia, medible y
simple de la situación de deserción definida como: dos años consecutivos sin
actividad. Esto es sin cursar ni aprobar finales.


a fin de dar el mejor apoyo a los alumnos de la Universidad y a la Universidad misma.


Objetivos Particulares
Desarrollar algoritmos de modelos descriptivos para la explicación de las variables relevantes al problema elegidocomo ejemplo: Estudiante que será docente.
Desarrollar algoritmos de modelos predictivos para evaluar algún aspecto particular del comportamiento actual de los
alumnos de la institución.
Obtener patrones y evaluar los mismos para crear las reglas y la Base de Conocimiento de un Sistema Experto,
mediante el resultado de los patrones obtenidos.
Desarrollo de un Sistema Experto para evaluar el comportamiento estudiantil a fin de dar soporte a la toma de
decisiones.


