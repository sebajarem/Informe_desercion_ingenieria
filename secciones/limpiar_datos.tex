\todo[inline, color=red]{reredactar y combinar con los notebooks de juoyter}

Limpieza
Como parte del proceso de limpieza de la base de datos se realizaron, entre otros:
\begin{itemize}
\item La unificación de los nombres de las materias ya que figuran con distintas denominaciones y a su vez son distintas a los nombres que figuran en el plan de la materia.
\item La restricción de los datos a los alumnos que tienen cursadas y finales en el periodo analizado. 
\item La corrección de las materias que tenían notas mayores a 10. 
\item Corrección de registros con diferentes errores de formato.
\item Eliminación de registros con errores groseros que no hay posibilidad para corregirlos.
\item Entre otras incoherencias (algunas ya mencionadas en los informes exploratorios y de calidad) \todorevisar{recordar poner esto en los informes exploratorios cuando analice los archivos de base}
\end{itemize}

El proceso de limpieza tiene como entrada las 3 tablas mencionadas en la sección de colección inicial de datos \ref{datos:raw}


