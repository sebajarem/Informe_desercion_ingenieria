\comentarioinvisible{formateo de los datos para dsps hacer analsis estadistico  descriptivo con dlookr, y algunos graficos. ojo! es una exploracion inicial. Hay descriptivos que con para la seccion de calidad de datos}

\subsubsection{Análisis estadísticos generales}

Si bien los trabajos de calidad de datos se muestran en la siguiente sección, es necesario hacer un mínimo formateo de los datos para poder hacer una exploración inicial.

Las nuevas tablas formateadas son las siguientes \ref{tab:tabla_Alumnos_Estadisticos_Categoricos},  :

\begin{table}[!h]

\caption{\label{tab:tabla_Alumnos_Estadisticos_Categoricos}Tabla Alumnos, valores mas frecuentes}
\centering
\fontsize{10}{12}\selectfont
\begin{tabular}[t]{llrrr}
\toprule
\rowcolor{black}  \multicolumn{1}{c}{\textcolor{white}{\textbf{variable}}} & \multicolumn{1}{c}{\textcolor{white}{\textbf{característica}}} & \multicolumn{1}{c}{\textcolor{white}{\textbf{frecuencia}}} & \multicolumn{1}{c}{\textcolor{white}{\textbf{frecuencia\_pct}}} & \multicolumn{1}{c}{\textcolor{white}{\textbf{rank}}}\\
\midrule
\rowcolor{gray!6}  Pais & Argentina & 8242 & 98.88 & 1\\
Pais & Perú & 33 & 0.40 & 2\\
\rowcolor{gray!6}  Pais & Bolivia & 22 & 0.26 & 3\\
Localidad & Ciudad Autónoma de Buenos Aires & 3767 & 45.19 & 1\\
\rowcolor{gray!6}  Localidad & Lomas de Zamora & 163 & 1.96 & 2\\
\addlinespace
Localidad & Lanús Oeste & 124 & 1.49 & 3\\
\rowcolor{gray!6}  Provincia & CABA / Capital Federal & 5744 & 68.91 & 1\\
Provincia & BUENOS AIRES & 2183 & 26.19 & 2\\
\rowcolor{gray!6}  Provincia & NA & 88 & 1.06 & 3\\
Estudios.Secundarios & Bachiller & 3557 & 42.68 & 1\\
\addlinespace
\rowcolor{gray!6}  Estudios.Secundarios & Técnico & 2139 & 25.66 & 2\\
Estudios.Secundarios & NA & 1089 & 13.07 & 3\\
\rowcolor{gray!6}  Estado.Civil & SOLTERO & 8182 & 98.16 & 1\\
Estado.Civil & CASADO & 118 & 1.42 & 2\\
\rowcolor{gray!6}  Estado.Civil & NA & 19 & 0.23 & 3\\
\bottomrule
\end{tabular}
\end{table}


De la tabla de los datos de Alumnos \ref{tab:tabla_Alumnos_Estadisticos_Categoricos} que son mas frecuentes a nivel global, podemos observar que no se hay valores que llamen la atención para el enfoque del estudio en cuestión. \\
\underline{En resumen:} La mayoría de los estudiantes son nacionales, viven cerca a la facultad de concurrencia y son solteros (lo cual es lógico sabiendo que la gran mayoría de los que empiezan la facultad lo hacen a una edad temprana). Este último dato nos hace sospechar que puede ser que la base de datos personales no se actualiza con frecuencia por lo que al manipular estos datos debe hacerse con cuidado.
La educación secundaria es equilibrada.
\\

\begin{table}[!h]

\caption{\label{tab:tabla_Cursadas_Estadisticos_Categoricos}Tabla Cursadas, valores más frecuentes}
\centering
\fontsize{10}{12}\selectfont
\begin{tabular}[t]{llrrr}
\toprule
\rowcolor{black}  \multicolumn{1}{c}{\textcolor{white}{\textbf{variable}}} & \multicolumn{1}{c}{\textcolor{white}{\textbf{característica}}} & \multicolumn{1}{c}{\textcolor{white}{\textbf{frecuencia}}} & \multicolumn{1}{c}{\textcolor{white}{\textbf{frecuencia\_pct}}} & \multicolumn{1}{c}{\textcolor{white}{\textbf{rank}}}\\
\midrule
\rowcolor{gray!6}  Sexo & M & 171335 & 85.75 & 1\\
Sexo & F & 28480 & 14.25 & 2\\
\rowcolor{gray!6}  Curso & K1024 & 3909 & 1.96 & 1\\
Curso & K1043 & 3801 & 1.90 & 2\\
\rowcolor{gray!6}  Curso & K1025 & 3748 & 1.88 & 3\\
\addlinespace
Materia & Análisis Matemático I & 18528 & 9.27 & 1\\
\rowcolor{gray!6}  Materia & Álgebra y Geometría Analítica & 17385 & 8.70 & 2\\
Materia & Matemática Discreta & 16715 & 8.37 & 3\\
\rowcolor{gray!6}  Departamento & SISTEMAS & 101844 & 50.97 & 1\\
Departamento & CS.BS. U.D.B. MATEMATICA & 48267 & 24.16 & 2\\
\addlinespace
\rowcolor{gray!6}  Departamento & CS.BS. U.D.B. FISICA & 14290 & 7.15 & 3\\
Modalidad & Anual & 90208 & 45.15 & 1\\
\rowcolor{gray!6}  Modalidad & Cuat 1/2 & 58496 & 29.28 & 2\\
Modalidad & Cuat 2/2 & 50233 & 25.14 & 3\\
\rowcolor{gray!6}  Turno & Mañana & 84893 & 42.49 & 1\\
\addlinespace
Turno & Noche & 76198 & 38.13 & 2\\
\rowcolor{gray!6}  Turno & Tarde & 38724 & 19.38 & 3\\
Tipo.de.aprobación & Firmo & 67253 & 33.66 & 1\\
\rowcolor{gray!6}  Tipo.de.aprobación & No Firmo & 49360 & 24.70 & 2\\
Tipo.de.aprobación & Libre & 37796 & 18.92 & 3\\
\addlinespace
\rowcolor{gray!6}  Descripción.de.recursada.regular & No Recurso & 152426 & 76.28 & 1\\
Descripción.de.recursada.regular & Recurso 1 Vez & 30665 & 15.35 & 2\\
\rowcolor{gray!6}  Descripción.de.recursada.regular & Recurso 2 Veces & 10057 & 5.03 & 3\\
Descripción.de.recursada.libre & No Recurso & 173493 & 86.83 & 1\\
\rowcolor{gray!6}  Descripción.de.recursada.libre & Recurso 1 Vez & 19345 & 9.68 & 2\\
\addlinespace
Descripción.de.recursada.libre & Recurso 2 Veces & 4603 & 2.30 & 3\\
\bottomrule
\end{tabular}
\end{table}

\begin{table}[!h]

\caption{\label{tab:tabla_Cursadas_Estadisticos_Numericos_1de2}Tabla Cursadas, valores más frecuentes}
\centering
\resizebox{\linewidth}{!}{
\fontsize{10}{12}\selectfont
\begin{tabular}[t]{lrrrrrrrrr}
\toprule
\rowcolor{black}  \multicolumn{1}{c}{\textcolor{white}{\textbf{variable}}} & \multicolumn{1}{c}{\textcolor{white}{\textbf{promedio}}} & \multicolumn{1}{c}{\textcolor{white}{\textbf{desvío}}} & \multicolumn{1}{c}{\textcolor{white}{\textbf{minimo}}} & \multicolumn{1}{c}{\textcolor{white}{\textbf{maximo}}} & \multicolumn{1}{c}{\textcolor{white}{\textbf{P05}}} & \multicolumn{1}{c}{\textcolor{white}{\textbf{Q1}}} & \multicolumn{1}{c}{\textcolor{white}{\textbf{mediana}}} & \multicolumn{1}{c}{\textcolor{white}{\textbf{Q3}}} & \multicolumn{1}{c}{\textcolor{white}{\textbf{P95}}}\\
\midrule
\rowcolor{gray!6}  Año.de.ingreso & 2011.49 & 2.60 & 2008 & 2017 & 2008 & 2009 & 2011 & 2014 & 2016\\
Año.de.nacimiento & 1991.42 & 3.95 & 1951 & 1999 & 1985 & 1990 & 1992 & 1994 & 1997\\
\rowcolor{gray!6}  Ciclo.Lectivo.de.Cursada & 2013.33 & 2.62 & 2008 & 2017 & 2009 & 2011 & 2014 & 2016 & 2017\\
Cantidad.de.veces.recursada.regular & 0.66 & 5.46 & 0 & 99 & 0 & 0 & 0 & 0 & 2\\
\rowcolor{gray!6}  Cantidad.de.veces.recursada.libre & 0.26 & 2.79 & 0 & 99 & 0 & 0 & 0 & 0 & 1\\
\bottomrule
\end{tabular}}
\end{table}

\begin{table}[!h]

\caption{\label{tab:tabla_Cursadas_Estadisticos_Numericos_2de2}Tabla Cursadas, valores más frecuentes}
\centering
\resizebox{\linewidth}{!}{
\fontsize{10}{12}\selectfont
\begin{tabular}[t]{lrrrrrr}
\toprule
\rowcolor{black}  \multicolumn{1}{c}{\textcolor{white}{\textbf{variable}}} & \multicolumn{1}{c}{\textcolor{white}{\textbf{ceros}}} & \multicolumn{1}{c}{\textcolor{white}{\textbf{ceros\_pct}}} & \multicolumn{1}{c}{\textcolor{white}{\textbf{negativos}}} & \multicolumn{1}{c}{\textcolor{white}{\textbf{negativos\_pct}}} & \multicolumn{1}{c}{\textcolor{white}{\textbf{outliers}}} & \multicolumn{1}{c}{\textcolor{white}{\textbf{outliers\_pct}}}\\
\midrule
\rowcolor{gray!6}  Año.de.ingreso & 0 & 0.00 & 0 & 0 & 0 & 0.00\\
Año.de.nacimiento & 0 & 0.00 & 0 & 0 & 7268 & 3.64\\
\rowcolor{gray!6}  Ciclo.Lectivo.de.Cursada & 0 & 0.00 & 0 & 0 & 0 & 0.00\\
Cantidad.de.veces.recursada.regular & 152426 & 76.28 & 0 & 0 & 47389 & 23.72\\
\rowcolor{gray!6}  Cantidad.de.veces.recursada.libre & 173493 & 86.83 & 0 & 0 & 26322 & 13.17\\
\bottomrule
\end{tabular}}
\end{table}


De las tablas de datos de Cursadas \ref{tab:tabla_Cursadas_Estadisticos_Categoricos}, \ref{tab:tabla_Cursadas_Estadisticos_Numericos_1de2} y \ref{tab:tabla_Cursadas_Estadisticos_Numericos_2de2} que son mas frecuentes a nivel global, podemos observar que un 85\% de los estudiantes son de género masculino, no hay una materia que se destaque por haber cursado mas que otras y las que se encuentran dentro de las primeras 3 son aquellas de niveles iniciales por lo que quizas pudiera ser la dificultad de los estudiantes de pasar el nivel secundario al nivel universitario al principio de la carrera. Los turnos preferibles son los de turno mañana y turno noche. La mayoría de los registros representan a materia cuyo alumno no ha recursado ni una sola vez.

\begin{table}[!h]

\caption{\label{tab:tabla_Finales_Estadisticos_Categoricos}Tabla Finales, valores más frecuentes}
\centering
\fontsize{10}{12}\selectfont
\begin{tabular}[t]{llrrr}
\toprule
\rowcolor{black}  \multicolumn{1}{c}{\textcolor{white}{\textbf{variable}}} & \multicolumn{1}{c}{\textcolor{white}{\textbf{característica}}} & \multicolumn{1}{c}{\textcolor{white}{\textbf{frecuencia}}} & \multicolumn{1}{c}{\textcolor{white}{\textbf{frecuencia\_pct}}} & \multicolumn{1}{c}{\textcolor{white}{\textbf{rank}}}\\
\midrule
\rowcolor{gray!6}  Materia & Química & 5325 & 6.39 & 1\\
Materia & Ingeniería y Sociedad & 4935 & 5.93 & 2\\
\rowcolor{gray!6}  Materia & Sistemas y Organizaciones & 4867 & 5.84 & 3\\
Aprobado & 1 & 66694 & 80.07 & 1\\
\rowcolor{gray!6}  Aprobado & 0 & 16597 & 19.93 & 2\\
\addlinespace
Promociono & 0 & 69270 & 83.17 & 1\\
\rowcolor{gray!6}  Promociono & 1 & 14021 & 16.83 & 2\\
\bottomrule
\end{tabular}
\end{table}

\begin{table}[!h]

\caption{\label{tab:tabla_Finales_Estadisticos_Numericos_1de2}Tabla Finales, valores más frecuentes}
\centering
\resizebox{\linewidth}{!}{
\fontsize{10}{12}\selectfont
\begin{tabular}[t]{lrrrrrrrrr}
\toprule
\rowcolor{black}  \multicolumn{1}{c}{\textcolor{white}{\textbf{variable}}} & \multicolumn{1}{c}{\textcolor{white}{\textbf{promedio}}} & \multicolumn{1}{c}{\textcolor{white}{\textbf{desvío}}} & \multicolumn{1}{c}{\textcolor{white}{\textbf{minimo}}} & \multicolumn{1}{c}{\textcolor{white}{\textbf{maximo}}} & \multicolumn{1}{c}{\textcolor{white}{\textbf{P05}}} & \multicolumn{1}{c}{\textcolor{white}{\textbf{Q1}}} & \multicolumn{1}{c}{\textcolor{white}{\textbf{mediana}}} & \multicolumn{1}{c}{\textcolor{white}{\textbf{Q3}}} & \multicolumn{1}{c}{\textcolor{white}{\textbf{P95}}}\\
\midrule
\rowcolor{gray!6}  Año & 2013.10 & 2.47 & 2008 & 2017 & 2009 & 2011 & 2013 & 2015 & 2017\\
Nota & 5.87 & 2.68 & 0 & 11 & 2 & 4 & 6 & 8 & 10\\
\bottomrule
\end{tabular}}
\end{table}

\begin{table}[!h]

\caption{\label{tab:tabla_Finales_Estadisticos_Numericos_2de2}Tabla Finales, valores más frecuentes}
\centering
\resizebox{\linewidth}{!}{
\fontsize{10}{12}\selectfont
\begin{tabular}[t]{lrrrrrr}
\toprule
\rowcolor{black}  \multicolumn{1}{c}{\textcolor{white}{\textbf{variable}}} & \multicolumn{1}{c}{\textcolor{white}{\textbf{ceros}}} & \multicolumn{1}{c}{\textcolor{white}{\textbf{ceros\_pct}}} & \multicolumn{1}{c}{\textcolor{white}{\textbf{negativos}}} & \multicolumn{1}{c}{\textcolor{white}{\textbf{negativos\_pct}}} & \multicolumn{1}{c}{\textcolor{white}{\textbf{outliers}}} & \multicolumn{1}{c}{\textcolor{white}{\textbf{outliers\_pct}}}\\
\midrule
\rowcolor{gray!6}  Año & 0 & 0.00 & 0 & 0 & 0 & 0\\
Nota & 1561 & 1.87 & 0 & 0 & 0 & 0\\
\bottomrule
\end{tabular}}
\end{table}


De la tabla de datos de Finales \ref{tab:tabla_Finales_Estadisticos_Categoricos}, \ref{tab:tabla_Finales_Estadisticos_Numericos_1de2} y \ref{tab:tabla_Finales_Estadisticos_Numericos_2de2} que son mas frecuentes a nivel global, podemos observar que existen errores en los datos como la nota máxima y que existen registros del año 2017 cuando este estudio se hace hasta el 2016.



