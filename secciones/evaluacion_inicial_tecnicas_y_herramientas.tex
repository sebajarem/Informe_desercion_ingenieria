%# modelos de desgaste, churn, 
\comentarioinvisible{técnicas de la bibliografia: regresión logística, redes bayesianas, arboles de decisión}


Como fue mencionado en \ref{seccion:dm_en_desercion}, los análisis mas apropiados desde el enfoque de la minería de datos para realizar una clasificación binaria del fenómeno de deserción son: El análisis de supervivencia y el de desgaste \cite{PredictModel-GarciaVellidoNebot}

\comentarioinvisible{
\todohacer{explciar brevemete las técnicas que se van a usar y porque son adecuadas}}

Las técnicas que se emplean en este análisis son varias. Dentro de ella se encuentran métodos no supervisados y supervisados.\\
Dentro de las técnicas no supervisada, se emplearán clusters (de varios tipos: kmeans, jerárquico, etc) para evaluar si existe tendencia al agrupamiento y luego corroborar si los grupos tienen características relacionadas con el target. \\
Por otro lado, dentro del grupo de las técnicas supervisadas, se emplean técnicas que son explicativas (árboles simples de decisión, regresión logística, LDA) y técnicas que no son explicativas (RandomForest, GradientBoosting, SVM) y que lo mas explicativo que pueden informar en algunos casos es la importancia de los predictores.\\

De forma complementaria, se realizan análisis de reducción de dimensión y análisis de influencia de variables para generar datasets alternativos mas eficientes del que se esta partiendo.\\

Cada técnica se explica brevemente en la sección donde se aplican los modelos.\ref{generacion_modelos}

