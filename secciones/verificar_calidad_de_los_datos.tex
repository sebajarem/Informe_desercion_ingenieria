\comentarioinvisible{analisis de calidad de datos. pueden aparecer: datos perdidoas, errores en lso datos, errores de medicion, incoherencias de codificacion, metadatos errorneos, eperdida de datos o nulos, etc}

\begin{table}[!h]

\caption{\label{tab:tabla_alumnos_calidad}Tabla Alumnos, valores unicos y nulos}
\centering
\resizebox{\linewidth}{!}{
\fontsize{10}{12}\selectfont
\begin{tabular}[t]{llrrrrrr}
\toprule
\rowcolor{black}  \multicolumn{1}{c}{\textcolor{white}{\textbf{variable}}} & \multicolumn{1}{c}{\textcolor{white}{\textbf{tipo}}} & \multicolumn{1}{c}{\textcolor{white}{\textbf{observaciones}}} & \multicolumn{1}{c}{\textcolor{white}{\textbf{observaciones\_pct}}} & \multicolumn{1}{c}{\textcolor{white}{\textbf{nulos}}} & \multicolumn{1}{c}{\textcolor{white}{\textbf{nulos\_pct}}} & \multicolumn{1}{c}{\textcolor{white}{\textbf{valores\_unicos}}} & \multicolumn{1}{c}{\textcolor{white}{\textbf{valores\_unicos\_pct}}}\\
\midrule
\rowcolor{gray!6}  Pais & character & 8332 & 99.96 & 3 & 0.04 & 19 & 0.23\\
Localidad & character & 8247 & 98.94 & 88 & 1.06 & 364 & 4.37\\
\rowcolor{gray!6}  Provincia & character & 8247 & 98.94 & 88 & 1.06 & 25 & 0.30\\
Estudios.Secundarios & character & 7246 & 86.93 & 1089 & 13.07 & 10 & 0.12\\
\rowcolor{gray!6}  Estado.Civil & character & 8316 & 99.77 & 19 & 0.23 & 4 & 0.05\\
\bottomrule
\end{tabular}}
\end{table}

\begin{table}[!h]

\caption{\label{tab:tabla_cursada_calidad}Tabla Cursada, valores únicos y nulos}
\centering
\resizebox{\linewidth}{!}{
\fontsize{10}{12}\selectfont
\begin{tabular}[t]{llrrrrrr}
\toprule
\rowcolor{black}  \multicolumn{1}{c}{\textcolor{white}{\textbf{variable}}} & \multicolumn{1}{c}{\textcolor{white}{\textbf{tipo}}} & \multicolumn{1}{c}{\textcolor{white}{\textbf{observaciones}}} & \multicolumn{1}{c}{\textcolor{white}{\textbf{observaciones\_pct}}} & \multicolumn{1}{c}{\textcolor{white}{\textbf{nulos}}} & \multicolumn{1}{c}{\textcolor{white}{\textbf{nulos\_pct}}} & \multicolumn{1}{c}{\textcolor{white}{\textbf{valores\_unicos}}} & \multicolumn{1}{c}{\textcolor{white}{\textbf{valores\_unicos\_pct}}}\\
\midrule
\rowcolor{gray!6}  Sexo & character & 199815 & 100.00 & 0 & 0.00 & 2 & 0.00\\
Año.de.ingreso & numeric & 199815 & 100.00 & 0 & 0.00 & 10 & 0.01\\
\rowcolor{gray!6}  Año.de.nacimiento & numeric & 199815 & 100.00 & 0 & 0.00 & 43 & 0.02\\
Curso & character & 199787 & 99.99 & 28 & 0.01 & 946 & 0.47\\
\rowcolor{gray!6}  Materia & character & 199815 & 100.00 & 0 & 0.00 & 104 & 0.05\\
\addlinespace
Departamento & character & 199815 & 100.00 & 0 & 0.00 & 7 & 0.00\\
\rowcolor{gray!6}  Modalidad & character & 199815 & 100.00 & 0 & 0.00 & 4 & 0.00\\
Turno & character & 199815 & 100.00 & 0 & 0.00 & 3 & 0.00\\
\rowcolor{gray!6}  Ciclo.Lectivo.de.Cursada & numeric & 199815 & 100.00 & 0 & 0.00 & 10 & 0.01\\
Tipo.de.aprobación & character & 199815 & 100.00 & 0 & 0.00 & 6 & 0.00\\
\addlinespace
\rowcolor{gray!6}  Cantidad.de.veces.recursada.regular & numeric & 199815 & 100.00 & 0 & 0.00 & 7 & 0.00\\
Descripción.de.recursada.regular & character & 199815 & 100.00 & 0 & 0.00 & 7 & 0.00\\
\rowcolor{gray!6}  Cantidad.de.veces.recursada.libre & numeric & 199815 & 100.00 & 0 & 0.00 & 7 & 0.00\\
Descripción.de.recursada.libre & character & 199815 & 100.00 & 0 & 0.00 & 7 & 0.00\\
\bottomrule
\end{tabular}}
\end{table}

\begin{table}[!h]

\caption{\label{tab:tabla_finales_calidad}Tabla Finales, valores unicos y nulos}
\centering
\resizebox{\linewidth}{!}{
\fontsize{10}{12}\selectfont
\begin{tabular}[t]{llrrrrrr}
\toprule
\rowcolor{black}  \multicolumn{1}{c}{\textcolor{white}{\textbf{variable}}} & \multicolumn{1}{c}{\textcolor{white}{\textbf{tipo}}} & \multicolumn{1}{c}{\textcolor{white}{\textbf{observaciones}}} & \multicolumn{1}{c}{\textcolor{white}{\textbf{observaciones\_pct}}} & \multicolumn{1}{c}{\textcolor{white}{\textbf{nulos}}} & \multicolumn{1}{c}{\textcolor{white}{\textbf{nulos\_pct}}} & \multicolumn{1}{c}{\textcolor{white}{\textbf{valores\_unicos}}} & \multicolumn{1}{c}{\textcolor{white}{\textbf{valores\_unicos\_pct}}}\\
\midrule
\rowcolor{gray!6}  Materia & character & 83291 & 100 & 0 & 0 & 129 & 0.15\\
Año & numeric & 83291 & 100 & 0 & 0 & 10 & 0.01\\
\rowcolor{gray!6}  Nota & numeric & 83291 & 100 & 0 & 0 & 12 & 0.01\\
Aprobado & character & 83291 & 100 & 0 & 0 & 2 & 0.00\\
\rowcolor{gray!6}  Promociono & character & 83291 & 100 & 0 & 0 & 2 & 0.00\\
\bottomrule
\end{tabular}}
\end{table}


Se puede observar que en la tabla Alumnos \ref{tab:tabla_alumnos_calidad} no hay mucha variedad en el contenido que tiene cada campo. El porcentaje de nulos en el campo de Estudios.Secundarios es el valor mas alto pero aceptable para realizar modelos.\\

En la tabla cursadas \ref{tab:tabla_cursada_calidad} el único campo que tiene valores nulos es el de cursos cuyo contenido es el código de la materia. Esto puede deberse a un error al completar la base pero es un dato que no puede faltar ya que se completa en las actas y libretas de estudiantes.\\

En la tabla de finales \ref{tab:tabla_finales_calidad} puede observarse que la cantidad de valores únicos de nombres de materias es demasiado elevado en realación a la cantidad de materias que compone un plan de carrera (alrededor de 50 materias), lo que nos quiere decir que hay materias cuyos nombres los han escrito de distintas formas y no hay un único valor y/o que aparecen materias de planes anteriores al analizado cuyas materias pueden ser parecidas pero que se llaman distintas.


\paragraph{\textbf{Conclusiones}}
La calidad de los datos inciales es relativamente buena y esto se debe a un preprocesamiento inicial realizado antes de comenzar este trabajo. Sin embargo, se encontraron algunas situaciones en la que los datos pueden no ser del todo confiables como posible información desactualizada en los datos personales. A su vez, se encontraron errores en algunos campos particulares con una simple inspección, como en el campo Notas que algunas excedían el máximo permitido (10) y registros que no corresponden al período de análisis (2008 a 2016), pero son fáciles de corregir.

