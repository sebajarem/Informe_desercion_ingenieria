Partiendo de las 3 tablas originales, la eliminación de variables, el agregado de los nuevos atributos y por último las transformaciones realizadas (todas vistas en las seciones anteriores), se genera un tablón principal. El mismo se llama \textbf{\textit{"baseline\_2009.csv"}}. \\

Dicho tablón también fue generado en el script \textit{"Giar 20-09.ipynb"} luego de hacer el preprocesamiento.

A continuación de presenta una muestra aleatoria pero invertiremos las filas por columnas para que entre en la hoja:

\begingroup\fontsize{10}{12}\selectfont

\begin{longtable}[t]{llll}
\caption{\label{tab:tabla_tablon_baseline}Ejemplo de Tablón}\\
\toprule
\rowcolor{black}  \multicolumn{1}{c}{\textcolor{white}{\textbf{variable}}} & \multicolumn{1}{c}{\textcolor{white}{\textbf{10000007}}} & \multicolumn{1}{c}{\textcolor{white}{\textbf{10000015}}} & \multicolumn{1}{c}{\textcolor{white}{\textbf{10000016}}}\\
\midrule
\endfirsthead
\caption[]{Ejemplo de Tablón \textit{(continued)}}\\
\toprule
\rowcolor{black}  \multicolumn{1}{c}{\textcolor{white}{\textbf{variable}}} & \multicolumn{1}{c}{\textcolor{white}{\textbf{10000007}}} & \multicolumn{1}{c}{\textcolor{white}{\textbf{10000015}}} & \multicolumn{1}{c}{\textcolor{white}{\textbf{10000016}}}\\
\midrule
\endhead
\
\endfoot
\bottomrule
\endlastfoot
\rowcolor{gray!6}  Aprobado & 2 & 14 & 5\\
Cantidad de veces recursada regular & 0 & 0 & 0\\
\rowcolor{gray!6}  Ciclo Lectivo de Cursada & 2014 & 2014 & 2014\\
Descripción de recursada regular\_No Recurso & 1 & 4 & 7\\
\rowcolor{gray!6}  Descripción de recursada regular\_Recurso 1 Vez & 0 & 0 & 0\\
\addlinespace
Descripción de recursada regular\_Recurso 2 Veces & 0 & 0 & 0\\
\rowcolor{gray!6}  Descripción de recursada regular\_Recurso 3 Veces & 0 & 0 & 0\\
Descripción de recursada regular\_Recurso 4 Veces & 0 & 0 & 0\\
\rowcolor{gray!6}  Descripción de recursada regular\_Recurso 5 Veces & 0 & 0 & 0\\
Descripción de recursada regular\_Recurso n Veces (>5) & 0 & 0 & 0\\
\addlinespace
\rowcolor{gray!6}  deserto & 1 & 1 & 1\\
edad al ingreso & 22 & 35 & 34\\
\rowcolor{gray!6}  EsTecnico & NA & NA & NA\\
noAprobado & 0 & 0 & 0\\
\rowcolor{gray!6}  Nota & 10 & 9 & 10\\
\addlinespace
Nota\_max\_prom & 10 & 9.92857142857143 & 10\\
\rowcolor{gray!6}  Promociono & 0 & 0 & 0\\
Sexo & M & M & M\\
\rowcolor{gray!6}  Tipo de aprobación\_Cambio Curso & 0 & 0 & 0\\
Tipo de aprobación\_Firmo & 0 & 1 & 0\\
\addlinespace
\rowcolor{gray!6}  Tipo de aprobación\_Libre & 0 & 3 & 6\\
Tipo de aprobación\_No Firmo & 1 & 1 & 1\\
\rowcolor{gray!6}  Tipo de aprobación\_Promociono & 0 & 0 & 0\\
Turno\_Mañana & 0 & 0 & 0\\
\rowcolor{gray!6}  Turno\_Noche & 1 & 4 & 6\\
\addlinespace
Turno\_Tarde & 0 & 1 & 1\\*
\end{longtable}
\endgroup{}





\todorevisar{vale la pena hacer estadisticos aca ? ahora son todos numericos (la mayoria)}



