\begin{itemize}
	\item 
	Los resultados son muy buenos y cumplen la expectativa inicial.
	\item 
	Se generaron modelos para distintos tipos de análisis y perfiles de usuarios distintos.
	\item 
	El proyecto se había iniciado hace un tiempo atrás y la información actualmente es antigua. Sin embargo, este trabajo indica que existe la posibilidad de realizar estos análisis y que sean productivos para los trabajadores sociales o para la misma facultad con el fin de tomar decisiones o realizar análisis mas profundos.
	\item 
	Al ser un trabajo cuya propuesta ya se había iniciado, estaban marcados algunos objetivos y datos, los cuales pueden ampliarse para una próxima versión.
	\item 
	Los modelos no supervisados como clusters y reducción de dimensionalidad (PCA y t-sne) no obtuvieron buenos resultados. Sin embargo no se ddescarta aplicarlos con otro tipo de preprocesamiento de la información original.
	\item 
	La información original tiene una gran cantidad de información. Es muy grande en comparación del tablón final con el cual se trabajó. Esto se debe a que muchos datos tuvieron que eliminarse por errores, incoherencias, no existía la posibildiad de consultar con un experto y la imputación no era viable en muchos casos. Además, se tomó la decisión de trabajar con información agrupada por lo que el tablón final resulta ser mucho mas pequeño. No se descarta trabajar con la información sin agrupar para realizar estudios mas detallados.
\end{itemize}
