%######################
%# importante
%# metodología de trabajo y que se va a hacer con este trabajo

En particular:
Se establecerá un protocolo para la obtención de datos de la base de datos de la facultad acordado con las
autoridades respectivas de manera de preservar el secreto estadístico (la identidad de los estudiantes cuyos datos serán
analizados). La cantidad de datos deberá ser representativa de la población donde se produce el fenómeno.
Se seleccionarán las herramientas más adecuadas al problema y a la base de datos a tratar.
Se realizarán experiencias y se desarrollarán modelos descriptivos y predictivos del comportamiento de los
estudiantes en lo referente al comportamiento seleccionado como ejemplo. Se sintonizarán y/o adecuarán los diferentes
algoritmos utilizados.
Se compararán los resultados obtenidos por los distintos algoritmos.
Con los patrones encontrados y un relevamiento de las variables pertinentes se propondrán las bases de
conocimiento para la construcción de un sistema experto.
Se buscará interacción con un experto u otro grupo del área educativa para ofrecerle los datos y plantear un trabajo
conjunto a futuro.
Se presentarán para publicar trabajos que muestren los resultados obtenidos en los ensayos experimentales, así
como también las apreciaciones teóricas respecto de los marcos conceptuales en los que se trabajó.