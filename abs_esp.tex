%\begin{center}
%\large \bf \runtitulo
%\end{center}
%\vspace{1cm}
\chapter*{\runtitulo}

\noindent El presente trabajo expone la investigación realizada sobre datos de estudiantes de la UTN \bnote{UTN} con el fin de tratar de identificar cuales son las características o patrones de comportamiento que corresponden a estudiantes con altas probabilidades de que en un futuro abandonen sus estudios, convirtiéndose en desertores de la carrera de ingeniería. El enfoque de análisis propuesto es mediante un análisis exploratorio inicial para la comprensión de los datos y posteriormente la aplicación de técnicas que se encuadran dentro del campo de la minería de datos. \textcolor{red}{Puede concluirse que es factible la detección temprana de posibles futuros estudiantes desertores habiendo obtenido un xx\% de casos acertados}. Por último, este proceso se realizó siguiendo la Metodología CRISP-DM lo que asegura la reproducibilidad de este trabajo y extensiva documentación de cada paso del proyecto.
\textcolor{red}{(aprox. 200 palabras)}.!!

\bigskip

\noindent\textbf{Palabras claves:} Minería de Datos, cluster, árbol de decisión, Gradient Boosting Machine (GBM), R, Python.


