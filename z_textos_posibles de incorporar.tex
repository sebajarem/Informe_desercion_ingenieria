En el caso de ejemplo se busca aportar herramientas para que los expertos en ciencias sociales y educativas vean facilitada su tarea de comprensión de la evolución de los estudiantes. Si bien existen iniciativas de características similares, en esta investigación se involucrará el empleo de herramientas de Inteligencia Artificial. De manera tal que sea posible encontrar y orientar en forma temprana a estos estudiantes, detectándolos a partir de sus patrones de comportamiento.


Se publicó un artículo titulado "Data mining en ciencias sociales"
en las II Jornadas de Ingeniería y Sociedad (JISO 2016)
organizadas en la ciudad de Puerto Madryn por la UTN - FRCH.
Joaquín Toranzo Calderón

\hyperref[crisp-dm]{https://www.ibm.com/support/knowledgecenter/es/SS3RA7_sub/modeler_crispdm_ddita/clementine/crisp_help/crisp_overview_container.html}




Dial-in ID: 344505
Dial-in Passcode: 6859
