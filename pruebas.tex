\section{pruebas}


si escribimos algo aca

\begin{table}[!h]

\caption{\label{tab:tabla_Dataset_Alumnos}Campos Tabla Alumnos}
\centering
\fontsize{10}{12}\selectfont
\begin{tabular}[t]{ll}
\toprule
\multicolumn{1}{c}{Dataset Alumnos} & \multicolumn{1}{c}{Observaciones: 8335} \\
\cmidrule(l{3pt}r{3pt}){1-1} \cmidrule(l{3pt}r{3pt}){2-2}
\rowcolor{black}  \multicolumn{1}{c}{\textcolor{white}{\textbf{variable}}} & \multicolumn{1}{c}{\textcolor{white}{\textbf{tipo}}}\\
\midrule
\rowcolor{gray!6}  Codigo.Alumno & integer\\
Pais & character\\
\rowcolor{gray!6}  Localidad & character\\
Provincia & character\\
\rowcolor{gray!6}  Estudios.Secundarios & character\\
\addlinespace
Estado.Civil & character\\
\bottomrule
\end{tabular}
\end{table}

\begin{table}[!h]

\caption{\label{tab:tabla_Dataset_Cursadas}Campos Tabla Cursadas}
\centering
\fontsize{10}{12}\selectfont
\begin{tabular}[t]{ll}
\toprule
\multicolumn{1}{c}{Dataset Cursadas} & \multicolumn{1}{c}{Observaciones: 199815} \\
\cmidrule(l{3pt}r{3pt}){1-1} \cmidrule(l{3pt}r{3pt}){2-2}
\rowcolor{black}  \multicolumn{1}{c}{\textcolor{white}{\textbf{variable}}} & \multicolumn{1}{c}{\textcolor{white}{\textbf{tipo}}}\\
\midrule
\rowcolor{gray!6}  Codigo.Alumno & double\\
Sexo & character\\
\rowcolor{gray!6}  Año.de.ingreso & double\\
Año.de.nacimiento & double\\
\rowcolor{gray!6}  Curso & character\\
\addlinespace
Materia & character\\
\rowcolor{gray!6}  Departamento & character\\
Modalidad & character\\
\rowcolor{gray!6}  Turno & character\\
Ciclo.Lectivo.de.Cursada & double\\
\addlinespace
\rowcolor{gray!6}  Tipo.de.aprobación & character\\
Cantidad.de.veces.recursada.regular & double\\
\rowcolor{gray!6}  Descripción.de.recursada.regular & character\\
Cantidad.de.veces.recursada.libre & double\\
\rowcolor{gray!6}  Descripción.de.recursada.libre & character\\
\bottomrule
\end{tabular}
\end{table}

\begin{table}[!h]

\caption{\label{tab:tabla_Dataset_Finales}Campos Tabla Finales}
\centering
\fontsize{10}{12}\selectfont
\begin{tabular}[t]{ll}
\toprule
\multicolumn{1}{c}{Dataset Finales} & \multicolumn{1}{c}{Observaciones: 83291} \\
\cmidrule(l{3pt}r{3pt}){1-1} \cmidrule(l{3pt}r{3pt}){2-2}
\rowcolor{black}  \multicolumn{1}{c}{\textcolor{white}{\textbf{variable}}} & \multicolumn{1}{c}{\textcolor{white}{\textbf{tipo}}}\\
\midrule
\rowcolor{gray!6}  Codigo.Alumno & character\\
Materia & character\\
\rowcolor{gray!6}  Año & character\\
Nota & character\\
\rowcolor{gray!6}  Aprobado & character\\
\addlinespace
Promociono & character\\
\bottomrule
\end{tabular}
\end{table}



\begin{table}[!h]

\caption{\label{tab:tabla_Alumnos_Estadisticos_Categoricos}Tabla Alumnos, valores mas frecuentes}
\centering
\fontsize{10}{12}\selectfont
\begin{tabular}[t]{llrrr}
\toprule
\rowcolor{black}  \multicolumn{1}{c}{\textcolor{white}{\textbf{variable}}} & \multicolumn{1}{c}{\textcolor{white}{\textbf{característica}}} & \multicolumn{1}{c}{\textcolor{white}{\textbf{frecuencia}}} & \multicolumn{1}{c}{\textcolor{white}{\textbf{frecuencia\_pct}}} & \multicolumn{1}{c}{\textcolor{white}{\textbf{rank}}}\\
\midrule
\rowcolor{gray!6}  Pais & Argentina & 8242 & 98.88 & 1\\
Pais & Perú & 33 & 0.40 & 2\\
\rowcolor{gray!6}  Pais & Bolivia & 22 & 0.26 & 3\\
Localidad & Ciudad Autónoma de Buenos Aires & 3767 & 45.19 & 1\\
\rowcolor{gray!6}  Localidad & Lomas de Zamora & 163 & 1.96 & 2\\
\addlinespace
Localidad & Lanús Oeste & 124 & 1.49 & 3\\
\rowcolor{gray!6}  Provincia & CABA / Capital Federal & 5744 & 68.91 & 1\\
Provincia & BUENOS AIRES & 2183 & 26.19 & 2\\
\rowcolor{gray!6}  Provincia & NA & 88 & 1.06 & 3\\
Estudios.Secundarios & Bachiller & 3557 & 42.68 & 1\\
\addlinespace
\rowcolor{gray!6}  Estudios.Secundarios & Técnico & 2139 & 25.66 & 2\\
Estudios.Secundarios & NA & 1089 & 13.07 & 3\\
\rowcolor{gray!6}  Estado.Civil & SOLTERO & 8182 & 98.16 & 1\\
Estado.Civil & CASADO & 118 & 1.42 & 2\\
\rowcolor{gray!6}  Estado.Civil & NA & 19 & 0.23 & 3\\
\bottomrule
\end{tabular}
\end{table}





\hypertarget{TOC}{}

\hypertarget{header}{}
\hypertarget{sample-document}{%
	\section{Sample Document}\label{sample-document}}

\hypertarget{sebastian-jaremczuk}{%
	\paragraph{Sebastian Jaremczuk}\label{sebastian-jaremczuk}}

\hypertarget{section}{%
	\paragraph{2020-03-31}\label{section}}

\hypertarget{titu}{}
\hypertarget{titu}{%
	\subsection{titu}\label{titu}}

\begin{lstlisting}[language=R]
print("notebook")
\end{lstlisting}

\begin{lstlisting}
## [1] "notebook"
\end{lstlisting}

\hypertarget{lo}{}
\hypertarget{lo}{%
	\section{lo}\label{lo}}

\begin{lstlisting}[language=R]
print("2ndo chunk")
\end{lstlisting}

\begin{lstlisting}
## [1] "2ndo chunk"
\end{lstlisting}


%%%%%%%%%%
%
%\definecolor{shadecolor}{gray}{0.9}
%
%%%%%
%
%\hypertarget{TOC}{}
%
%\hypertarget{header}{}
%\hypertarget{sample-document}{%
%	\section{Sample Document}\label{sample-document}}
%
%\hypertarget{sebastian-jaremczuk}{%
%	\paragraph{Sebastian Jaremczuk}\label{sebastian-jaremczuk}}
%
%\hypertarget{section}{%
%	\paragraph{2020-03-31}\label{section}}
%
%\hypertarget{titu}{}
%\hypertarget{titu}{%
%	\subsection{titu}\label{titu}}
%
%\begin{shaded}
%	\begin{Highlighting}[]
%		\KeywordTok{print}\NormalTok{(}\StringTok{"notebook"}\NormalTok{)}
%	\end{Highlighting}
%\end{shaded}
%
%\begin{verbatim}
%## [1] "notebook"
%\end{verbatim}
%
%\hypertarget{lo}{}
%\hypertarget{lo}{%
%	\section{lo}\label{lo}}
%
%\begin{Shaded}
%	\begin{Highlighting}[]
%		\KeywordTok{print}\NormalTok{(}\StringTok{"2ndo chunk"}\NormalTok{)}
%	\end{Highlighting}
%\end{Shaded}
%
%\begin{verbatim}
%## [1] "2ndo chunk"
%\end{verbatim}


\begin{longtable}[]{@{\extracolsep{\fill}}ll@{}}
\toprule
variable & tipo\tabularnewline
\midrule
\endhead
codigo.alumno & character\tabularnewline
tipo\_de\_aprobacion\_libre & double\tabularnewline
tipo\_de\_aprobacion\_cambio\_curso & double\tabularnewline
tipo\_de\_aprobacion\_promociono & double\tabularnewline
edad\_al\_ingreso & double\tabularnewline
tipo\_de\_aprobacion\_no\_firmo & double\tabularnewline
ciclo\_lectivo\_de\_cursada & double\tabularnewline
tipo\_de\_aprobacion\_firmo & double\tabularnewline
cantidad\_de\_veces\_resursada\_regular & double\tabularnewline
descripción\_de\_recursada\_regular\_No\_Recurso & double\tabularnewline
descripción\_de\_recursada\_regular\_Recurso1vez & double\tabularnewline
descripción\_de\_recursada\_regular\_Recurso2vez & double\tabularnewline
descripción\_de\_recursada\_regular\_Recurso3vez & double\tabularnewline
descripción\_de\_recursada\_regular\_Recurso4vez & double\tabularnewline
descripción\_de\_recursada\_regular\_Recurso5vez & double\tabularnewline
descripción\_de\_recursada\_regular\_RecursoNveces &
double\tabularnewline
EsTecnico & character\tabularnewline
deserto & character\tabularnewline
Sexo & character\tabularnewline
Turno\_Tarde & double\tabularnewline
Turno\_Noche & double\tabularnewline
Turno\_Mañana & double\tabularnewline
Aprobado & double\tabularnewline
Promociono & double\tabularnewline
noAprobado & double\tabularnewline
Nota & double\tabularnewline
Nota\_max\_prom & double\tabularnewline
\bottomrule
\end{longtable}

\begin{lstlisting}[language=R]
# estadisticos_calidad(datos = baseline,
#                      variables_excluir = c("codigo.alumno"),
#                      generar_xlsx = F)
# 
# estadisticos_categoricos(datos = baseline,
#                          variables_excluir = c("codigo.alumno"),
#                          generar_xlsx = F,
#                          top = 3)
# 
# estadisticos_numericos(datos = baseline,
#                        variables_excluir = c("codigo.alumno"),
#                        generar_xlsx = F)
\end{lstlisting}













